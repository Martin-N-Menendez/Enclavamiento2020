% Chapter Template

\chapter{Conclusiones} % Main chapter title

\label{Chapter5} % Change X to a consecutive number; for referencing this chapter elsewhere, use \ref{ChapterX}


En este capítulo se presentan las conclusiones obtenidas y los logros alcanzados, además de un resumen de los elementos del proyecto que no se incluyeron en la memoria.

%----------------------------------------------------------------------------------------

\section{Resultados obtenidos}
	
	A lo largo de todo el proyecto se han visitado una gran cantidad estaciones, talleres, oficinas y áreas de pruebas. Se ha establecido relación con decenas de profesionales del área que, con sus conocimientos y experiencias, han sabido aportar a la resolución del problema abordado. Esto, combinando tanto el esfuerzo propio como la ayuda de los integrantes del CONICET-GICSAFe, permitió implementar de forma exitosa un prototipo del sistema de enclavamiento electrónico.
	
	Los conocimientos en el área de las FPGAs sirvieron para materializar los diseños planteados, llegando a obtener no solo un sistema seguro y funcional, sino también escalable, y además la implementación de un sistema que brinda la generación automática de la solución.
	
	%El haber trabajado en conjunto en otros proyectos de CONICET-GICSAFe a la par del desarrollo del sistema de enclavamientos permitió tomar conciencia de su importancia como sistema crítico y abre las puertas a una tercera etapa próxima a iniciarse. %Con una base sólida de conocimientos, expectativas mucho mas altas y vínculos humanos y profesionales, pero con la confianza y el respaldo que otorgan los avances concretados en esta etapa que llega a su fin.

	En el transcurso del proyecto se han alcanzado los siguientes logros:
	
	\begin{itemize}
		\item Diseño e implementación de un algoritmo analizador de redes ferroviarias, por lo que pueden analizarse la mayoría de las topologías de las redes ferroviarias argentinas.
		\item Diseño e implementación de un generador de código en VHDL basado en un grafo ferroviario, por lo que pueden implementarse enclavamientos electrónicos cualquiera sea la locación.
		\item Diseño e implementación de un generador de tramas para comandar la plataforma FPGA desde Python, que facilita el testeo de los sistemas generados.
		\item Publicación de artículos en IEEE Latin America y el Congreso Argentino de Sistemas Embebidos 2019 \cite{IEEE_LAT}.
		\item Se completó con éxito una beca de Maestría UBACyT.
		\item Se obtuvo una beca de doctorado en desarrollo estratégico de CONICET 2020-2025 en la que se continuará el desarrollo, abordando los aspectos relativos a redundancia y diversidad, y además se espera probar la solución desarrollada en una locación real.
		%\item Desarrollo del primer sistema ferroviario crítico en el marco del CONICET-GICSAFe.
	\end{itemize}
	
\section{Próximos pasos}

	La planificación del proyecto contempló desde un inicio que sería un trabajo de dos años a ser realizado en conjunto entre la Especialización y la Maestría de Sistemas Embebidos. Sin embargo, el mencionado incremento en la complejidad del sistema abrió las puertas a continuar el proyecto durante el doctorado. Por lo tanto, se mencionan a continuación los pasos a seguir para el próximo tramo del proyecto:	
	
	\begin{itemize}
		\item Optimización del analizador de grafos ferroviarios desarrollado, con el fin de abarcar topologías mas complejas que las presentadas.
		\item Integración con la interfaz gráfica deserrollada en UTN Facultad Regional Haedo \cite{UTN}.
		\item Realización de pruebas en paralelo con la estación Olivos, gracias al desarrollo del sistema modular del Mg. Ing. Lucas Dórdolo \cite{Lucas}.
		\item Ampliación de la batería de ensayos, para abarcar mas casos en topologías mas complejas.
		\item Automatización de los ensayos, para evitar realizar pruebas de fuerza bruta exhaustivamente.
		\item Ampliación del generador de tablas de enclavamiento, para abarcar mas condiciones de funcionamiento.
		%\item Corrección del funcionamiento de las barreras.
		%\item Elaboración de un análisis formal del sistema de enclavamientos para redes ferroviarias arbitrarias.
		\item Aplicación de técnicas de redundancia y diversidad por votación para aumentar la seguridad del sistema.
		\item Realización de pruebas en una locación real.
		\item Determinación de los niveles RAMS alcanzados para así establecer si el sistema diseñado puede ser utilizado en aplicaciones reales, o en su defecto cuáles son los aspectos en los que debe ser mejorado.
	\end{itemize}
