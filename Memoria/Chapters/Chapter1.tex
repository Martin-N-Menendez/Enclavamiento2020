% Chapter 1

\chapter{Introducción General} % Main chapter title

\label{Chapter1} % For referencing the chapter elsewhere, use \ref{Chapter1} 
\label{IntroGeneral}

%----------------------------------------------------------------------------------------

% Define some commands to keep the formatting separated from the content 
\newcommand{\keyword}[1]{\textbf{#1}}
\newcommand{\tabhead}[1]{\textbf{#1}}
\newcommand{\code}[1]{\texttt{#1}}
\newcommand{\file}[1]{\texttt{\bfseries#1}}
\newcommand{\option}[1]{\texttt{\itshape#1}}
\newcommand{\grados}{$^{\circ}$}

%----------------------------------------------------------------------------------------

En este capítulo se presentan el contexto del proyecto, los sistemas ferroviarios, las distintas tecnologías de los sistemas de enclavamientos y los objetivos a cumplir.

%----------------------------------------------------------------------------------------
		
	\section{Contexto y motivación}
	
		En la actualidad, el sistema ferroviario argentino se encuentra deteriorado y desactualizado. Mientras que otras naciones han progresado en la migración de sus sistemas de transporte de cargas y pasajeros añadiendo electrónica de última generación, Argentina continúa utilizando mecanismos diseñados a principios o mediados del siglo XX.
				
		Esto repercute negativamente en la seguridad que dichos sistemas pueden brindar a los pasajeros, conductores, peatones y automovilistas. A la vez, ocurre que los sistemas electrónicos para la seguridad vial de trenes y subtes son importados y muy costosos, además de estar monopolizados por menos de una docena de empresas en todo el mundo. Un sistema de enclavamiento como el que se aborda en este trabajo puede costar entre 5 y 10 millones de dólares\citep{SIEMENS} y se requieren varias decenas de estos sistemas sólo para la zona urbana de la Ciudad Autónoma de Buenos Aires.
		
		Las empresas que fabrican sistemas de enclavamiento brindan muy poca información técnica sobre sus desarrollos y la mayoría de las herramientas que utilizan son de uso corporativo. No obstante, deben regirse bajo normativas como las que la Unión Europea ha establecido desde 2004. Las tres normas principales son EN-50126\citep{EN50126} (ciclo de vida), EN-50128\citep{EN50128} (técnicas de software) y EN-50129\citep{EN50129} (técnicas de hardware). %Entre otras cuestiones, se busca la integridad de la seguridad.	
		
		Los sistemas de enclavamiento son sistemas críticos. Esto implica que una falla puede poner en peligro cientos de vidas humanas y/o costosas inversiones. Por lo tanto, se deben cumplir estrictos parámetros de fiabilidad, disponibilidad, mantenibilidad y seguridad (RAMS, del inglés \textit{Reliability}, \textit{Availability}, \textit{Mantenibility} \textit{and} \textit{Safety}), durante todo el ciclo de vida.
		
		En este contexto, en 2015 se creó el CONICET-GICSAFe, cuyas siglas corresponden a Grupo de Investigación en Calidad y Seguridad de las Aplicaciones Ferroviarias, conformado por docentes e investigadores de una decena de universidades e instituciones públicas argentinas\citep{GICSAFE}. El grupo desarrolla sistemas electrónicos e informáticos para aplicaciones ferroviarias relacionadas con la seguridad, a partir de la generación de un prototipo funcional y la documentación correspondiente que luego se transfiere en su totalidad a los clientes. En particular, esta metodología se utiliza en este trabajo con Trenes Argentinos, que es la Sociedad del Estado que opera las líneas Roca, Sarmiento, Mitre, San Martín y Belgrano Sur, entre otras. Luego el cliente puede fabricar el sistema diseñado por el GICSAFe o licitar su fabricación, así como modificar el sistema de acuerdo con sus necesidades.   
	
		En el marco de la Especialización de Sistemas Embebidos, desde julio de 2018 se tuvieron reuniones con diferentes funcionarios y profesionales de Trenes Argentinos. En particular los encuentros se desarrollaron con la Gerencia de Ingeniería, Gerencia de Seguridad Operacional, Subgerencia de Desarrollo y Normas Técnicas, Subgerencia de Transporte, Gerencia de Señalamiento, entre otros, de los cuales surgió el interés en el desarrollo del presente proyecto. De dichas visitas y otras posteriores se obtuvo la totalidad de las fotografías incluidas en esta memoria.
			 	
		El desarrollo del sistema involucra muchas áreas distintas: procesamiento, comunicación, replicación del sistema para otras topologías, testing, etc. Un primer prototipo se desarrolló a finales de 2018 y actualmente se culminó una segunda versión para la Maestría en Sistemas Embebidos. La amplia variedad de topologías existentes obligó a abandonar la metodología de desarrollo ad hoc de cada sistema y fue necesario abordar el diseño de forma general, para topologías genéricas, lo que adiciona aún más trabajo al proyecto.		
		
		Por esa razón, durante la Maestría en Sistemas Embebidos, en el año 2019, se comenzó a seguir un estrategia orientada a obtener una solución integral para cualquier locación, generando automáticamente el código y los testbenchs involucrados. A la vez que miembros de CONICET-GICSAFe pertenecientes a UTN-Haedo comenzaron el desarrollo de un front-end gráfico y su correspondiente simulador en tiempo real, en vistas a ser integrado al proyecto en el mediano plazo.
	
		Hacia finales de 2019 se iniciaron reuniones con miembros de la Comisión Nacional de Energía Atómica (CNEA), para integrar el proyecto en sus plataformas de hardware ampliamente testeadas en el ámbito de los sistemas críticos. Además del intercambio de conocimiento y la puesta en común de estrategias a utilizar, se aprovechó todo lo posible la amplia experiencia que ellos brindaron a GICSAFe desde fines del año pasado a la fecha.
		
	\section{Propuesta de solución}	
		
		En el marco de la Especialización de Sistemas Embebidos se implementó en 2018 un sistema similar al de la Figura \ref{fig:CESE_1}. El mismo contaba con una cantidad limitada de tramos de vías y bifurcaciones que implicaba una cantidad acotada de circuitos lógicos.% complicados pero posibles de obtener y ensayar.
		
		La implementación se hacía de forma ad hoc, bloque a bloque y conectándolos de igual manera uno por uno. Diseñar los ensayos demoraba días o semanas y era necesario revisar varias veces las pocas especificaciones que se tenían para corroborar un correcto funcionamiento del sistema.	
		
		\begin{figure}[htbp!]
			\centering
			\includegraphics[scale=.5]{./Figures/Grafo_VHDL}
			\caption{Ejemplo de implementación de topología sencilla.}
			\label{fig:CESE_1}
		\end{figure}	
		
		\vspace{5cm}
		
		Durante la especialización fue una constante que la locación en la cual se basaba el desarrollo del sistema cambiase abruptamente por otra diferente, representada en la figura \ref{fig:CESE_2}. Por lo que, aun manteniendo los mismos conceptos iniciales, el alcance y los requerimientos cambiaban completamente y se tenía que desechar gran parte de lo construido, tanto del sistema como de los tests, que ya no tenían sentido para el nuevo sistema. El rediseñar de todo de cero fue algo común durante esta etapa del proyecto.
		
		\begin{figure}[htbp!]
			\centering
			\includegraphics[scale=.5]{./Figures/Grafo_VHDL_B}
			\caption{Ejemplo de implementación de topología compleja.}
			\label{fig:CESE_2}
		\end{figure}
			
		Una mayor cantidad de elementos ferroviarios implica una densidad mayor de circuitos lógicos, y estos a su vez repercuten en un crecimiento exponencial de la complejidad del desarrollo, mayor dificultad para implementar los ensayos y mayores chances de errores al extrapolar conceptos aún inmaduros a sistemas de mayor tamaño.
		
		La solución a proponer no podía quedar sujeta a una única topología, ya que aun culminando el proyecto se corría el riesgo de que el tiempo empleado fuese desperdiciado si se cambiaban los requerimientos de forma tan abrupta. Es por eso que se añadió como objetivo de este trabajo el desarrollo de una herramienta capaz de generar automáticamente la solución electrónica de un sistema de enclavamiento ferroviario, a partir de una representación matemática única de cada topología. En la figura \ref{fig:Generacion} se puede visualizar el proceso, en el cual diversas topologías son procesadas e implementadas en una plataforma FPGA.
		
		\begin{figure}[htbp!]
			\centering
			\includegraphics[scale=.5]{./Figures/Generacion}
			\caption{Proceso de generación automática de la solución electrónica.}
			\label{fig:Generacion}
		\end{figure}
		
		De esta manera, la solución parte de cualquier red ferroviaria, representada mediante un grafo y un analizador de redes ferroviarias (desarrollado también en este trabajo), detecta qué función cumplirá cada elemento de la red y determina cuántos semáforos, de cuántos aspectos, en qué orientación y en qué nodo deben situarse para que el sistema sea seguro. A continuación, en base a los semáforos insertados el analizador calcula todas las rutas posibles que admite esa red y genera automáticamente la solución electrónica para ser implementada mediante una FPGA.
	
		Para poder entender mejor el funcionamiento del sistema es necesario introducir conceptos propios del mundo ferroviario y sus componentes involucrados, que se describen en la sección siguiente.

	\section{Elementos del señalamiento ferroviario}
	
		La función del señalamiento ferroviario es evitar las colisiones entre trenes y los descarrilamientos. A continuación se describen los diferentes elementos que componente el señalamiento y que fueron modelados durante este trabajo.
		
		\subsection{Vías}
			
			Las vías férreas (figura \ref{fig:Via_eclisa}) consisten en el elemento esencial de la infraestructura ferroviaria y conforman el sitio por el cual se desplazan los trenes. Se encuentran separadas por una distancia fija que se mide desde sus caras internas y se denomina trocha.
			
			\begin{figure}[htbp!]
				\centering
				\includegraphics[scale=.07]{./Figures/Tramo_via}
				\caption{Tramo de vía ferroviaria.}
				\label{fig:Via_eclisa}
			\end{figure}	
			
			\vspace{7cm}
			
			Las vías se dividen en secciones y por seguridad se establece que cada sección puede contener solo una formación por vez. Las mismas pueden tener largos variables en zonas urbanas de entre 500 a 1000 metros en zonas rurales. 
			
			Cada vía puede ser clasificada en dos grupos: vías ascendentes o vías descendentes. Las ascendentes son aquellas por las que los trenes circulan únicamente en la dirección del kilometraje en sentido creciente. Las descendentes son aquellas por las que los trenes circulan únicamente en la dirección del kilometraje en sentido decreciente\citep{RITO}. El kilómetro 0 es la estación principal de la línea ferroviaria, como por ejemplo: Plaza Constitución (para la línea Roca), Once de septiembre (para la línea Sarmiento) o Retiro (para las líneas Mitre y San Martín).
			
			Existen vías de maniobra que pueden ser tanto ascendentes como descendentes. Estas vinculan, mediante un cambio de vías, una sección ascendente con otra descendente, en la cual los trenes deben circular a una velocidad reducida.
			
		\subsection{Semáforos ferroviarios}
			
			El sistema de enclavamientos utiliza los semáforos ferroviarios para indicarle al conductor del tren si puede o no acceder al próximo tramo de vías y a qué velocidad se le permite circular; esto, por medio del color del semáforo, denominado aspecto. A diferencia de los semáforos vehiculares, en los que cada color es alternado por otro de la secuencia rojo-amarillo-verde en función del tiempo, los semáforos ferroviarios cambian su aspecto en función de los eventos de los tramos siguientes.
					
			En la figura \ref{fig:Sem_3Aspectos} se presenta un esquema de señales de tres aspectos, que es el tipo de semáforo que se utiliza en la gran mayoría de las líneas ferroviarias.
						 	
			 \begin{figure}[htbp!]
				\centering
				\includegraphics[scale=.35]{./Figures/Sem3}
				\caption{(a) Semáforo de tres aspectos\\(b) Semáforo doble de tres aspectos (Estación Olivos).}
				\label{fig:Sem_3Aspectos}
			\end{figure}
			
			Otra diferencia fundamental es que no todos los semáforos ferroviarios poseen tres aspectos. Los semáforos de maniobras constan de solo dos, amarillo (precaución) y rojo (prohibido avanzar), y algunas líneas como la Línea Roca utilizan semáforos de cuatro aspectos.	
			
			En la figura \ref{fig:Sem_2Aspectos} se visualizan los semáforos de dos aspectos. Se utilizan en cambios de vías donde, por su peligrosidad, solo se podrían permitir aspectos rojos y amarillos.
			 
			 \begin{figure}[htbp!]
				\centering
				\includegraphics[scale=.33]{./Figures/Sem2}
				\caption{(a) Semáforo de dos aspectos\\(b) Semáforos de cruce de vías (Estación Lavallol).}
				\label{fig:Sem_2Aspectos}
			\end{figure}				
			
			Los semáforos de cuatro aspectos son utilizados en la Línea Roca y poseen un doble amarillo antes del amarillo simple, para permitir así tramos de vías mas cortos en forma segura. Como no son objeto de estudio del presente trabajo, no serán explicados aquí.
		
		\subsection{Circuito de vías}
		
			Para poder determinar si un tramo de la vía se encuentra ocupado o libre se utilizan los circuitos de vías. Estos constituyen componentes electrónicos que imponen una tensión conocida entre los rieles, y cuando un tren se posiciona sobre esa sección provoca un cortocircuito que es detectado por el circuito. En la figura \ref{fig:Ocupacion} se ejemplifica la ocupación de las secciones por una formación (modelada con un 0) y la ausencia de formación (modelada con un 1).
			
			\begin{figure}[h]
				\centering
				\includegraphics[scale=.4]{./Figures/Ocupacion}
				\caption{Ocupación de las secciones de vías.}
				\label{fig:Ocupacion}
			\end{figure}
			
			\vspace{7cm}
			
			Si el tramo de vía no tiene ninguna formación ocupándolo, el señalamiento indicará un aspecto verde o amarillo según el estado de ocupación del tramo siguiente. Si la formación ocupa la sección, el señalamiento cambiará su aspecto a rojo para indicar que no puede ingresar ninguna otra formación, a fin de evitar colisiones. Por seguridad también se establecerá a rojo el semáforo anterior y a amarillo el anterior a este (doble recubrimiento), tal como se ilustra en la figura \ref{fig:Recubrimiento}.			
					
			\begin{figure}[h]
				\centering
				\includegraphics[scale=.4]{./Figures/Recubrimiento}
				\caption{Estado de los aspectos ferroviarios según la ubicación del tren.}
				\label{fig:Recubrimiento}
			\end{figure}			
			
			Si la alimentación es interrumpida o si el cableado sufre alguna falla, entonces el sistema asumirá que hay un tren en las vías y los semáforos se pondrán en aspecto rojo para que las formaciones cercanas detengan su marcha y las barreras de los pasos a nivel desciendan. A este principio se lo denomina \textit{fail-safe}\footnote{\textit{fail-safe}: falla segura.}. Es decir, si por alguna razón algo falla, el sistema adopta la condición mas restrictiva, mitigando la posibilidad de una situación peligrosa. 		
			
		\subsection{Pasos a nivel}
		
			La intersección de una ruta vehicular o peatonal con la vía férrea se denomina paso a nivel. El sistema de control de la barrera mantiene el brazo de esta en alto para permitir la circulación vehicular, como se puede ver en la figura \ref{fig:Paso_a_nivel}. Si un tren ocupa las secciones amarillas de la figura \ref{fig:Paso_a_nivel} se desenergiza la barrera y comienza a descender el brazo por efecto de la gravedad. Cuando se ocupen las secciones azules de la figura \ref{fig:Paso_a_nivel}, entonces se accionará la alarma sonora para alertar a los peatones que deben permanecer en el laberinto contiguo a la vía, cuya función es forzar a los peatones a mirar a ambos lados antes de cruzar el paso a nivel.			
			
			\begin{figure}[h!]
				\centering
				\includegraphics[scale=0.4]{./Figures/Paso a nivel}
				\caption{Pasos a nivel vehicular y peatonal sobre vía férrea.}
				\label{fig:Paso_a_nivel}
			\end{figure}
			%\vspace{5cm}
			
			Solo cuando la barrera baja, el tren tiene permitido avanzar sobre el cruce, siendo el paso a nivel un sector de altísimo riesgo.
			
			Al desocuparse las las secciones amarillas, la barrera vuelve a energizarse y se sitúa en estado alto nuevamente, a la espera de otro tren para reiniciar el proceso descripto.
						
			Se debe destacar que el mismo proceso de descenso de la barrera ocurrirá si esta se desenergiza por una falla eléctrica y/o pérdida de alimentación. Es decir, el sistema asumirá el estado mas seguro ante cualquiera de los mencionados fallos, siguiendo el principio de falla segura.
		
		\subsection{Máquina de cambios}
			
			Una máquina de cambios (figura \ref{fig:Cambio}) es un mecanismo utilizado para permitir el paso de las formaciones de una vía a una ramificación del recorrido principal. Esto se realiza mediante el movimiento de la aguja del cambio (riel móvil) hacia su respectiva contraaguja (riel fijo) hasta obtener un adecuado acoplamiento que permita la circulación de la formación.
			
			\begin{figure}[h!]
				\centering
				\includegraphics[scale=.06]{./Figures/Cambio}
				\caption{Máquina de cambios de Lavallol (Línea Roca).}
				\label{fig:Cambio}
			\end{figure} 
			
			\vspace{7cm}
			
			En la figura \ref{fig:Cambios_2} se muestra el cambio de vía de la estación Matheu de la Línea Mitre. Se observa que según sea la posición de la máquina de cambios, el tren puede continuar en la misma vía o hacer el cambio a la otra vía.
			
			\begin{figure}[h!]
				\centering
				\includegraphics[scale=.1]{./Figures/Cambios_2}
				\caption{Cambio de vías de estación Matheu (Línea Mitre).}
				\label{fig:Cambios_2}
			\end{figure} 
		
			En la figura \ref{fig:Cambios} se muestran las posiciones que puede adoptar el cambio. En la posición normal los trenes pueden circular de forma directa, en paralelo, por la vía principal en sentidos opuestos. En la posición reversa, en cambio, se permite el intercambio de trenes de una rama principal a otra en sentido opuesto o a una ramificación secundaria de la red.
			
			\begin{figure}[h!]
				\centering
				\includegraphics[scale=.45]{./Figures/Cambios}
				\caption{Posiciones normal e inversa del cambio.}
				\label{fig:Cambios}
			\end{figure} 	
					
	\section{Sistema de enclavamientos}

		A modo de ejemplo se ilustra en la figura \ref{fig:Bypass} un sistema de cambios en una vía simple con \emph{bypass}. Este permite que dos formaciones puedan cruzarse en sentidos opuestos sin colisionar.
	
		\begin{figure}[h!]
			\centering
			\includegraphics[scale=.45]{./Figures/Bypass_2}
			\caption{Vía simple con \textit{bypass}.}
			\label{fig:Bypass}
		\end{figure}
		%\vspace{5cm}
			
		Para evitar la colisión, se requiere un control seguro que evite que las formaciones avancen hacia secciones ya ocupadas por otras. También debe evitar que las formaciones avancen sobre los cambios cuando estos aún no han terminado de posicionarse en su lugar, lo que provocaría descarrillamientos. A este control se lo denomina sistema de enclavamiento y en definitiva impide que se produzcan las configuraciones no seguras y controla los semáforos que habilitan o no los itinerarios de las formaciones.		
					 
	 	Una falla en un enclavamiento puede poner en peligro cientos de vidas humanas y generar gastos considerables. Por lo tanto, en el diseño del sistema de enclavamiento se deben cumplir estrictos parámetros de fiabilidad, disponibilidad, mantenibilidad y seguridad (RAMS).
	 	
	 	Lamentablemente, los sistemas de enclavamientos en Argentina son en su mayoría mecánicos, de comienzos del siglo XX, y otra cantidad considerable son electromecánicos, de mas de 40 años de antiguedad. Muchos de ellos ya han agotado su vida útil y deben ser reemplazados. Otros, en cambio, han estado en desuso por años y necesitan ser repuestos, pero solo una docena de empresas en el mundo realizan el diseño del sistema y los costos para un bypass simple rondan las decenas de millones de dólares. Por esto, es importante contar con sistemas electrónicos de diseño y fabricación nacional.

		Además, existen diferentes lugares donde aún resta instalar este tipo de sistemas, por lo que su implementación constituye una necesidad real para el desarrollo de la infraestructura ferroviaria de señalamiento en Argentina.		
	
	\section{Tipos de enclavamientos}
		
		A continuación se presentan distintas tecnologías de implementación de enclavamientos en orden cronológico de invención.
		
		\subsection{Enclavamientos mecánicos}
			
			A comienzos del siglo XX se implementaron los sistemas de enclavamientos mediante soluciones mecánicas. Utilizaban palancas como las que se visualizan en la figura \ref{fig:Mecanico} para comandar los cambios de vías y semáforos.
	
			\begin{figure}[htbp!]
				\centering
				\includegraphics[scale=.27]{./Figures/Mecanico}
				\caption{Sistema enclavamiento mecánico en la estación de Chascomús, hoy convertida en museo.}
				\label{fig:Mecanico}
			\end{figure}

			Una vez que se constituye una configuración de posiciones de palancas que habilitan un trayecto, estas quedan 'enclavadas' mecánicamente. Es decir, su posición se bloquea y no es físicamente posible cambiarla. A medida que se van moviendo ciertas palancas, las demás que pudieran representar situaciones no seguras quedan enclavadas, y solo se pueden mover aquellas cuyo accionamiento representa una situación segura. De esa manera se garantiza que no se generarán configuraciones tales que las formaciones colisionen entre sí.
			
			Las tecnologías mas modernas heredaron el término ''enclavamiento'', aunque ya no se tengan palancas enclavadas en posiciones fijas.
		
		\subsection{Enclavamientos electromecánicos}
			
			A mediados del siglo XX se desarrolló el sistema de enclavamiento electromecánico. Su funcionamiento se basa en relés (figura \ref{fig:Reles}) y circuitos de vía, de forma tal de poder detectar la presencia de un tren y comandar tanto las señales como las barreras de los pasos a nivel.
	
			%Se visitó un sistema de enclavamientos en la estación Lavallol (figura \ref{fig:Reles})  de la Línea Roca. La misma es una sala refrigerada, de gran tamaño, con varios bastidores que contienen cientos de relés.	
			
			\begin{figure}[htbp!]
				\centering
				\includegraphics[scale=.08]{./Figures/Reles}
				\caption{Bastidor de relés de estación Lavallol (Línea Roca).}
				\label{fig:Reles}
			\end{figure}
		
			\vspace{7cm}
			
			Los sistemas de enclavamiento electromecánicos son comandados por un operario mediante un panel de control (figura \ref{fig:Electromecanico}). El operario solicita al sistema de enclavamiento las rutas que el conductor ferroviario necesita para circular. El sistema permitirá solo la operación de cambio de vías seguras. En caso contrario, se tendrán las salidas ''enclavadas'' y el sistema de enclavamiento impedirá mediante los semáforos el avance de la formación hasta que pueda realizarse el cambio en forma segura.
		
			\begin{figure}[h!]
				\centering
				\includegraphics[scale=.27]{./Figures/Electromecanico}
				\caption{Panel de control enclavamientos - Central Lavallol.}
				\label{fig:Electromecanico}
			\end{figure}
		
			%\vspace{5cm}
			
			%El sistema debe evitar colisiones en los trenes, pero no siempre se tiene toda la red con circuitos de vías. Por lo tanto es el operario el que debe recordar a que sectores envió tal o cual formación. Lo que implica una enorme responsabilidad que recae en el operario y el factor de error humano aumenta conforme el personal se encuentre mas cansado o distraído por alguna razón.
		
		\subsection{Enclavamientos electrónicos}
		\label{sec:Redundancia}	
			%La redundancia es obligatoria en sistemas electrónicos críticos para obtener un sistema tolerante a fallas, operando ante eventos no previstos u erróneos.
			
			El sistema de enclavamiento moderno es electrónico y debe incluir redundancia de hardware para lograr niveles RAMS adecuados. Pueden utilizarse, por ejemplo, estrategias de \emph{2 en 2} o sistemas de votación \emph{2 de 3}. En la figura \ref{fig:Redundancia} se presenta un sistema con redundancia \emph{2 de 3}, el mismo tendrá una salida correcta siempre que se tenga a lo sumo un fallo simultáneo\citep{REDUNDANCIA}. 
			
			También se ilustra en la Figura \ref{fig:Redundancia} el concepto de diversidad de plataformas de hardware (representados en diferente color). Para mitigar fallos comunes a una misma plataforma de hardware es una buena estrategia el utilizar sistemas de diferentes marcas o proveedores, para minimizar esta problemática. De esta forma, el resultado global es un sistema inmune a fallas singulares. No obstante, es vulnerable a fallas simultáneas de dos componentes, pero su probabilidad es mínima al ser de diferentes tecnologías u orígenes.
			
			\begin{figure}[h]
				\centering
				\includegraphics[scale=.45]{./Figures/Redundancia}
				\caption{Redundancia por votación 2 de 3.}
				\label{fig:Redundancia}
			\end{figure}
			
			En este trabajo se implementó un sistema de enclavamiento electrónico, como se explicará en el Capítulo \ref{Chapter3}.
	 
	 		%\vspace{5cm}
	 		
%	\section{Tabla de enclavamientos}
%	\label{Tabla_enclavamientos}
%	\label{sec:Semi}	
%		La forma de diseñar y presentar los enclavamientos es mediante tablas de enclavamiento. La figura \ref{fig:Matheu_a} presenta la tabla correspondiente a la estación Matheu de la Línea Mitre.
%	
%		\begin{figure}[htbp!]
%			\centering
%			%\includegraphics[scale=.4]{./Figures/Matheu_a}
%			\caption{Tabla de enclavamientos mecánicos (Matheu-Mitre).}
%			\label{fig:Matheu_a}
%		\end{figure}
%		
%		La tabla define en que posiciones se enclavan las palancas de la segunda columna, respecto de las condiciones de la columna 'normal' e 'invertida'.
%		
%		Por ejemplo, en el caso de la palanca 1, que corresponde a la señal de distancia 2 y 3, si las palancas 2 y 3 se encuentran en posición invertida, entonces la palanca 1 se enclavará en invertida sin posibilidad de pasarla a normal hasta que las otras dos palancas reviertan su estado.
%			
%		En este trabajo se utiliza una forma alternativa de representar las tablas de enclavamiento, que se introduce a continuación, mediante un ejemplo en base a la figura \ref{fig:Esquema_vacio}.		
%		
%		\begin{figure}[htbp!]
%			\centering
%			%\includegraphics[scale=.3]{./Figures/Tabla_vacia}
%			\caption{Esquema de ejemplo en el que las señales no se han asignado.}
%			\label{fig:Esquema_vacio}
%		\end{figure}
%		
%		\vspace{5cm}
%		
%		Se supone una topología de vías dobles con un cambio de vías entre ambas para realizar maniobras, dos pasos peatonales con campanilla (paso a nivel 1 y 3) y un paso a nivel vehicular con barrera automática (paso a nivel 2). 
%		
%		Además se presentan los semáforos de tres aspectos correspondientemente numerados de forma creciente en el sentido de circulación marcado para cada trayecto (impares para la vía ascendente y pares de la descendente). Se ilustran además, los semáforos de dos aspectos (10 y 11) para las maniobras de cambio de vías.
%		
%		Para comenzar el razonamiento, todo el señalamiento se muestra apagado (gris) ya que se busca determinar en qué aspecto debe estar cada señal. Supóngase que una formación ocupa el circuito de vía 8, detenido antes del semáforo 6 y se dispone a completar el trayecto descendente hacia la sección 6 de la vía.%, correspondiente a la estación. %Cabe destacar que el cambio de vía en este caso se toma en reversa, por lo que se denomina una 'maniobra de talón'.
%		
%		%Si nos encontramos en el modo semiautomático, el operario debe solicitar al sistema de enclavamiento cada ruta que desee que la formación recorra. Una ruta siempre se define de un semáforo al semáforo inmediatamente consecutivo. Por lo tanto mientras la ruta no esté aprobada los semáforos deben estar en aspecto rojo y como la aprobación requiere la presencia de una formación ocupando el circuito de vía inicial de la misma, se concluye que la vía ascendente tendrá todos los semáforos en rojo en ausencia de formación alguna.
%		
%		Ya que la formación usará la vía principal, los semáforos 10 y 11 correspondientes al cambio de vías también estarán a rojo, al igual que los semáforos 2 y 8 que no involucran a la ruta a solicitar. La ruta en cuestión se definirá entre los semáforos 6 (inicial) y el 4 (final). Para comenzar el trayecto la formación necesita permiso para acceder a los tramos de vías siguientes, para lo cual el semáforo 6 deberá estar en aspecto verde. Para finalizar el trayecto se necesita que el semáforo 4 se encuentre en rojo. Por seguridad se pusieron a rojo el semáforo 8 (para evitar colisiones de formaciones que vengan antes de la formación de ejemplo) y el semáforo 2 (por si la formación no llega a frenar ante el semáforo 4).
%		
%		El paso a nivel número 3 debería estar a peligro desde que la formación ocupo el circuito de vía 10, el número 2 debería tener su barrera baja por la misma razón y el número 1 estará a peligro tan pronto la formación ocupe el circuito de vía 6b. Por lo tanto los tres pasos a nivel se encuentran a peligro y no deberán permitir el cruce ni de vehículos ni de peatones mientras la ruta esté en ejecución. Durante todo el trayecto también se deberá garantizar que la máquina de cambios se encuentre en posición 'normal'. De esta forma se tiene la configuración que se muestra en la figura \ref{fig:Esquema_lleno}.
%		
%		\begin{figure}[htbp!]
%			\centering
%			%\includegraphics[scale=.3]{./Figures/Tabla_llena}
%			\caption{Esquema de ejemplo en el que las señales se han asignado.}
%			\label{fig:Esquema_lleno}
%		\end{figure}
%		
%		La Tabla \ref{tab:Ejemplo_semi} representa la situación descripta, siendo:
%		
%		\begin{table}[htbp!]
%			\centering
%			\caption[Modelo de Tabla de enclavamientos semiautomáticos]{Modelo de Tabla de enclavamientos semiautomáticos}
%			\resizebox{\textwidth}{!}{  
%			\begin{tabular}{c|cccccc|ccccc|cccc|cc|c|ccc}    
%				\toprule
%				& \multicolumn{6}{c|}{Circuitos de vía (DES)} & \multicolumn{5}{c|}{Semáforos ASC} & \multicolumn{4}{c|}{Semáforos DES} & \multicolumn{2}{c|}{Cambio} & M & \multicolumn{3}{c}{PaN} \\
%				\textbf{Rutas} 	 & 2 & 4 & 6a & 6b & 8 & 10 & 1 & 3 & 5 & 7 & 9 & 2 & 4 & 6 & 8 & 10 & 11 & - & 1 & 2 & 3  \\
%				\midrule
%				$\text{Ruta}$ &O&O&O&O&X&O&R&R&R&R&R&R&R&V&R&R&R&N&B&B&B\\ 
%				%\bottomrule
%				\hline
%			\end{tabular}
%			}
%			\label{tab:Ejemplo_semi}
%		\end{table}
%		
%		\begin{itemize}
%			\item O: Circuito de vía que debe permanecer libre.
%			\item X: Circuito de vía que se encuentra ocupado por una formación.
%			\item R: Semáforo rojo.
%			\item A: Semáforo amarillo.
%			\item V: Semáforo verde.
%			\item -: \emph{Don't care}, no es relevante.
%			\item N: Máquina de cambios en posición normal.
%			\item R: Máquina de cambios en posición reversa.
%			\item A: Barrera alta.
%			\item B: Barrera baja.
%		\end{itemize}								
%		
%		Repitiendo el razonamiento se pueden generar las demás filas de la tabla para cualquier ruta entre dos semáforos. Solo se puede tener un tren en cada vía a la vez: uno para la vía ascendente y otro para la vía descendente.
%		
%%		En el caso del modo automático la construcción de la tabla es muy similar, pero cambia el criterio para los semáforos.
%%		
%%		Por otro lado, en el modo automático los semáforos por defecto se encuentran en verde y cada tren que ocupa la vía genera hacia atrás una estela de protección denominada doble recubrimiento[REF] a peligro como puede verse en la figura \ref{fig:Doble}.
%%		
%%		\begin{figure}[htbp!]
%%			\centering
%%			\includegraphics[scale=.33]{./Figures/Doble}
%%			\caption{Doble recubrimiento a peligro}
%%			\label{fig:Doble}
%%		\end{figure}
%%				
%%		Dicho recubrimiento consiste en dos semáforos a rojo (semáforo 7 y 9) para proteger el tramo que ocupa la formación y el anterior (circuitos de vía 11 y 9), además de un semáforo amarillo de precaución (semáforo 5) para avisar a la formación anterior que la próxima señal será roja y debe disminuir su marcha.					
%%		
%%		En esta modalidad no se pueden realizar cambios de vías, por lo que la máquina de cambios estará en posición 'normal' y los semáforos de cambios 10 y 11 estarán en rojo. Además todos los pasos a nivel estarán a peligro menos el paso a nivel número 1, que al estar alejado de la formación que sigue avanzando en sentido ascendente, dejará de sonar su campanilla.
%%		
%%		La Tabla \ref{tab:Ejemplo_auto} presenta el ejemplo anterior pero para un sistema automático con una formación ocupando el circuito de vía 4.
%%		
%%		\begin{table}[htbp!]
%%			\centering
%%			\caption[Modelo de Tabla de enclavamientos semiautomáticos]{Modelo de Tabla de enclavamientos semiautomáticos}
%%			\resizebox{\textwidth}{!}{  
%%			\begin{tabular}{c|cccccc|ccccc|cccc|cc|c|ccc}    
%%				\toprule
%%				& \multicolumn{6}{c|}{Circuitos de vía(ASC)} & \multicolumn{5}{c|}{Semáforos ASC} & \multicolumn{4}{c|}{Semáforos DES} & \multicolumn{2}{c|}{Cambio} & M & \multicolumn{3}{c}{PaN} \\
%%				\textbf{Rutas} 	 & 1 & 3 & 5 & 7 & 9 & 11 & 1 & 3 & 5 & 7 & 9 & 2 & 4 & 6 & 8 & 10 & 11 &  & 1 & 2 & 3  \\
%%				\midrule
%%				$\text{Ruta}$ &-&-&-&O&O&X&V&V&A&R&R&V&V&V&V&R&R&N&A&B&B\\ 
%%				%\bottomrule
%%				\hline
%%			\end{tabular}
%%			}
%%			\label{tab:Ejemplo_auto}
%%		\end{table}
%%	
%%		Los circuitos de vía 1,3 y5 figuran como \emph{don't care} porque queda fuera del doble recubrimiento y podría ser ocupado por otra formación. Además el semáforo para acceder a dicho tramo se establece en verde, lo que permite el acceso.
%		
	\section{Objetivos}
	
		El objetivo de este proyecto fue el diseño, implementación y realización de pruebas funcionales de un prototipo de sistema electrónico de enclavamiento, sobre un kit de desarrollo de FPGA. 
		
		Se procuró además estudiar las tecnologías para implementar metodologías orientadas a mejorar los niveles RAMS del sistema, de acuerdo con el estado del arte en sistemas ferroviarios altamente críticos. 
		
		Teniendo la experiencia acumulada del trabajo realizado en la Especialización de Sistemas Embebidos, se puso especial énfasis en la automatización del proceso, para poder satisfacer las necesidades de cualquier zona elegida. 

%		Para lo cual se definieron los siguientes requerimientos funcionales:
%		
%		\begin{itemize}
%			\item Cantidad de entradas: alrededor de 40
%			\item Tipo de entrada: contactos secos
%			\item Clase de entradas: ocupación de vía, señalamiento, estado de la máquina de cambios o posición de barreras del paso a nivel.
%			\item Tiempo máximo entre cambios: 1 minuto
%			\item Topología: vía doble con estación, cruce y tres pasos a nivel
%			\item Distancia mínima entre formaciones: doble recubrimiento
%			\item Tiempo mínimo para recorrer el tramo:
%			\item Tiempo máximo para recorrer el tramo:
%			\item Cantidad de trenes circulando en simultáneo: hasta 2 en modo semiautomático y hasta cuatro en modo automático.
%			\item Cantidad de coches por formación: 6
%		\end{itemize}

	
%----------------------------------------------------------------------------------------






