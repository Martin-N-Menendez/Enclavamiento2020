\chapter{Diseño e Implementación} % Main chapter title

\label{Chapter3} % Change X to a consecutive number; for referencing this chapter elsewhere, use \ref{ChapterX}

\definecolor{mygreen}{rgb}{0,0.6,0}
\definecolor{mygray}{rgb}{0.5,0.5,0.5}
\definecolor{mymauve}{rgb}{0.58,0,0.82}

%%%%%%%%%%%%%%%%%%%%%%%%%%%%%%%%%%%%%%%%%%%%%%%%%%%%%%%%%%%%%%%%%%%%%%%%%%%%%
% parámetros para configurar el formato del código en los entornos lstlisting
%%%%%%%%%%%%%%%%%%%%%%%%%%%%%%%%%%%%%%%%%%%%%%%%%%%%%%%%%%%%%%%%%%%%%%%%%%%%%
\lstset{ %
  backgroundcolor=\color{white},   % choose the background color; you must add \usepackage{color} or \usepackage{xcolor}
  basicstyle=\footnotesize,        % the size of the fonts that are used for the code
  breakatwhitespace=false,         % sets if automatic breaks should only happen at whitespace
  breaklines=true,                 % sets automatic line breaking
  captionpos=b,                    % sets the caption-position to bottom
  commentstyle=\color{mygreen},    % comment style
  deletekeywords={...},            % if you want to delete keywords from the given language
  %escapeinside={\%*}{*)},          % if you want to add LaTeX within your code
  %extendedchars=true,              % lets you use non-ASCII characters; for 8-bits encodings only, does not work with UTF-8
  %frame=single,	                % adds a frame around the code
  keepspaces=true,                 % keeps spaces in text, useful for keeping indentation of code (possibly needs columns=flexible)
  keywordstyle=\color{blue},       % keyword style
  language=[ANSI]C,                % the language of the code
  %otherkeywords={*,...},           % if you want to add more keywords to the set
  numbers=left,                    % where to put the line-numbers; possible values are (none, left, right)
  numbersep=5pt,                   % how far the line-numbers are from the code
  numberstyle=\tiny\color{mygray}, % the style that is used for the line-numbers
  rulecolor=\color{black},         % if not set, the frame-color may be changed on line-breaks within not-black text (e.g. comments (green here))
  showspaces=false,                % show spaces everywhere adding particular underscores; it overrides 'showstringspaces'
  showstringspaces=false,          % underline spaces within strings only
  showtabs=false,                  % show tabs within strings adding particular underscores
  stepnumber=1,                    % the step between two line-numbers. If it's 1, each line will be numbered
  stringstyle=\color{mymauve},     % string literal style
  tabsize=2,	                   % sets default tabsize to 2 spaces
  title=\lstname,                  % show the filename of files included with \lstinputlisting; also try caption instead of title
  morecomment=[s]{/*}{*/}
}


%----------------------------------------------------------------------------------------
%	SECTION 1
%----------------------------------------------------------------------------------------
%\section{Análisis del software}
% 
%La idea de esta sección es resaltar los problemas encontrados, los criterios utilizados y la justificación de las decisiones que se hayan tomado.
%
%Se puede agregar código o pseudocódigo dentro de un entorno lstlisting con el siguiente código:
%
%\begin{verbatim}
%\begin{lstlisting}[caption= "un epígrafe descriptivo"]
%	las líneas de código irían aquí...
%\end{lstlisting}
%\end{verbatim}
%
%A modo de ejemplo:
%
%\begin{lstlisting}[label=cod:vControl,caption=Pseudocódigo del lazo principal de control.]  % Start your code-block
%
%#define MAX_SENSOR_NUMBER 3
%#define MAX_ALARM_NUMBER  6
%#define MAX_ACTUATOR_NUMBER 6
%
%uint32_t sensorValue[MAX_SENSOR_NUMBER];		
%FunctionalState alarmControl[MAX_ALARM_NUMBER];	//ENABLE or DISABLE
%state_t alarmState[MAX_ALARM_NUMBER];						//ON or OFF
%state_t actuatorState[MAX_ACTUATOR_NUMBER];			//ON or OFF
%
%void vControl() {
%
%	initGlobalVariables();
%	
%	period = 500 ms;
%		
%	while(1) {
%
%		ticks = xTaskGetTickCount();
%		
%		updateSensors();
%		
%		updateAlarms();
%		
%		controlActuators();
%		
%		vTaskDelayUntil(&ticks, period);
%	}
%}
%\end{lstlisting}

%Es este capítulo se presentarán las topologías básicas del sistema ferroviario y los dos enfoques de resolución del proyecto, con sus ventajas y desventajas antes de abordar la implementación de la solución elegida.

En este capítulo se presentarán las decisiones de diseño adoptadas para concretar el desarrollo del trabajo. Además de describir en forma genérica los módulos necesarios tanto del sistema de enclavamiento como de los bloques auxiliares para concretar una comunicación exitosa entre el sistema y el exterior.


\section{Consideraciones generales}

En el presente trabajo se optó por implementar el sistema bajo el enfoque funcional. Asumiendo que sus ventajas son mucho mas fuertes que sus desventajas y el análisis de los resultados obtenidos fueron mucho mas provechosos que los de su contraparte funcional.

\section{Generación automática}

	\begin{figure}[h]
	\centering
		\includegraphics[scale=.3]{./Figures/Generacion}
		\caption{Generación automática del código VHDL}
		\label{fig:hola}
	\end{figure}

\section{Módulo de nodos}

	\begin{figure}[h]
	\centering
	%\includegraphics[scale=.3]{./Figures/Redundancia}
		\caption{HOLA}
		\label{fig:hola}
	\end{figure}
	\improvement{Incluir figura}	
	
\section{Módulo de cambios}

	\begin{figure}[h]
	\centering
	%\includegraphics[scale=.3]{./Figures/Redundancia}
		\caption{HOLA}
		\label{fig:hola}
	\end{figure}
	\improvement{Incluir figura}	
	
\section{Módulo de red}

	\begin{figure}[h]
	\centering
	%\includegraphics[scale=.3]{./Figures/Redundancia}
		\caption{HOLA}
		\label{fig:hola}
	\end{figure}
	\improvement{Incluir figura}	
	
\section{Módulos de adaptación a enclavamiento}

	Explicacion 
	 
	\subsection{Módulo separador}
		Explicacion 
		
		\begin{figure}[h]
		\centering
			\includegraphics[scale=.3]{./Figures/FSMD-Separador}
			\caption{Diagrama de estados finitos digitales del módulo separador}
			\label{fig:FSMD_Separador}
		\end{figure}
		
		Explicacion 
		
	\subsection{Módulo mediador}
	
		Explicacion 
		
		\begin{figure}[h]
		\centering
			\includegraphics[scale=.3]{./Figures/FSMD-Mediador}
			\caption{Diagrama de estados finitos digitales del módulo mediador}
			\label{fig:FSMD_Mediador}
		\end{figure}
		
		Explicacion 
		
\section{Módulos de procesamiento de tramas}

	Explicacion 
	
	\subsection{Módulo detector}
	
		Explicacion 
		
		\begin{figure}[h]
		\centering
			\includegraphics[scale=.3]{./Figures/FSMD-Detector}
			\caption{Diagrama de estados finitos digitales del módulo detector}
			\label{fig:FSMD_Detector}
		\end{figure}
		
		Explicacion 
		
	\subsection{Módulo registro}
	
		Explicacion 
		
		\begin{figure}[h]
		\centering
			\includegraphics[scale=.3]{./Figures/FSMD-Registro}
			\caption{Diagrama de estados finitos digitales del módulo registro}
			\label{fig:FSMD_Registro}
		\end{figure}

		Explicacion 
		
	\subsection{Módulo selector}
	
		Explicacion 
		
		\begin{figure}[h]
		\centering
			%\includegraphics[scale=.3]{./Figures/FSMD_Selector}
			\caption{Diagrama de estados finitos digitales del módulo selector}
			\label{fig:FSMD_Selector}
		\end{figure}
		\improvement{Incluir figura}	
		
		Explicacion 
		
\section{Módulo de comunicación UART}

		\begin{figure}[h]
		\centering
		%\includegraphics[scale=.3]{./Figures/FSMD_UART}
			\caption{Diagrama de estados finitos digitales del módulo UART}
			\label{fig:FSMD_UART}
		\end{figure}
		\improvement{Incluir figura}

\section{Interfaz de comunicación Python}

		\begin{figure}[h]
		\centering
		%\includegraphics[scale=.3]{./Figures/Redundancia}
			\caption{HOLA}
			\label{fig:hola}
		\end{figure}
		\improvement{Incluir figura}